\documentclass{article}
\usepackage[utf8]{inputenc}

\title{Matrix-chain Multiplication}
\author{Manuel Serna-Aguilera}
\date{}

\usepackage{natbib}
\usepackage{graphicx}
\usepackage{clrscode3e}
\usepackage{amsmath}

\setlength{\parindent}{0pt}

\begin{document}

\maketitle

\section*{Problem}
Given sequence of matrices

\begin{equation*}
    A_1 \times A_2 \times A_3 \times A_4,
\end{equation*}

with dimensions

\begin{equation*}
    A_{5\times4} \times A_{4\times6} \times A_{6\times2} \times A_{2\times7},
\end{equation*}

we want to minimize the cost of multiplying the matrices. 
\\ \\
\subsection*{Finding Cost}
Given two matrices $A$ and $B$, to find the cost or number of scalar multiplications of the matrix product $A \times B = C$, multiply the dimension of the rows and columns of $A$ with the dimension of the columns of $B$. For example, given 

\begin{equation*}
    A_{5 \times 4} \times B_{4 \times 3} = C_{5 \times 3}
\end{equation*}

The number of scalar multiplications would be: $5 \cdot 4 \cdot 3 = 60$. Using a bottom-up approach, we start with single matrices, then move up to multiply pairs, and then sets of three, and so on, to get the cost of the entire sequence.
\\ \\
Now, create tables $m$ and $s$ with row $i$ and column $j$,
\begin{itemize}
    \item $m$: at $m[i, j]$, the minimum cost is stored when multiplying matrices from $A_i$ to $A_j$. We want to get to $m[1, 4]$, since we have $4$ matrices.
    
    \item $s$: at $s[i, j]$, row $i$ will give the sequence of the optimal ordering; basically, it tells us how to parenthesize the sequence of matrices from $i$ to $j$.
\end{itemize}

\newpage

%-------------------------------------------------
% m and s tables--initial
%-------------------------------------------------
\begin{center}
\label*{$m$}
\begin{tabular}{ c|c|c|c|c| }
    % Label columns
    \multicolumn{1}{c}{}
    & \multicolumn{1}{c}{1}
    & \multicolumn{1}{c}{2}
    & \multicolumn{1}{c}{3}
    & \multicolumn{1}{c}{4}\\

    % Label rows
    \cline{2-5}
        1 &  &  &  &  \\
    \cline{2-5}
        2 &  &  &  &  \\
    \cline{2-5}
        3 &  &  &  &  \\
    \cline{2-5}
        4 &  &  &  &  \\
    \cline{2-5}
\end{tabular}
\end{center}

\begin{center}
\label*{$s$}
\begin{tabular}{ c|c|c|c|c| }
    % Label columns
    \multicolumn{1}{c}{}
    & \multicolumn{1}{c}{1}
    & \multicolumn{1}{c}{2}
    & \multicolumn{1}{c}{3}
    & \multicolumn{1}{c}{4}\\

    % Label rows
    \cline{2-5}
        1 &  &  &  &  \\
    \cline{2-5}
        2 &  &  &  &  \\
    \cline{2-5}
        3 &  &  &  &  \\
    \cline{2-5}
        4 &  &  &  &  \\
    \cline{2-5}
\end{tabular}
\end{center}

For the procedure, we only need to fill out half of the table along the diagonal going top-left to bottom-right, as cases below are symmetrical.
\\ \\
%-------------------------------------------------
% Diagonal i = j
%-------------------------------------------------
Start at $m[i, j]$, where $i=j$.
\begin{itemize}
    \item The cost is zero, just having one matrix by itself would not cost anything to multiply. This is trivial, but for this diagonal, fill in all zeros.
\end{itemize}

\begin{center}
\label*{$m$}
\begin{tabular}{ c|c|c|c|c| }
    % Label columns
    \multicolumn{1}{c}{}
    & \multicolumn{1}{c}{1}
    & \multicolumn{1}{c}{2}
    & \multicolumn{1}{c}{3}
    & \multicolumn{1}{c}{4}\\

    % Label rows
    \cline{2-5}
        1 & 0 &  &  &  \\
    \cline{2-5}
        2 &  & 0 &  &  \\
    \cline{2-5}
        3 &  &  & 0 &  \\
    \cline{2-5}
        4 &  &  &  & 0 \\
    \cline{2-5}
\end{tabular}
\end{center}

%-------------------------------------------------
% Diagonal i = j + 1
%-------------------------------------------------
Move up to the next diagonal.
\begin{itemize}
    \item $m[1, 2]$: multiply matrices $A_1$ and $A_2$.\\
    $(A_{5 \times 4}) \times A_{4 \times 6}$ gives a cost of $5 \cdot 4 \cdot 6 = 120$. \\
    Thus, $m[1, 2] = 120$.\\
    Place value of 1 in $s[1, 2]$ since we split the sequence at $A_1$.
    
    \item $m[2, 3]$: multiply matrices $A_2$ and $A_3$.\\
    $(A_{4 \times 6}) \times A_{6 \times 2}$ gives a cost of $4 \cdot 6 \cdot 2 = 48$. \\
    Thus, $m[2, 3] = 48$.\\
    Place value of 2 in $s[2, 3]$ since we split the sequence at $A_2$.
    
    \item $m[3, 4]$: multiply matrices $A_3$ and $A_4$.\\
    $(A_{6 \times 2}) \times A_{2 \times 7}$ gives a cost of $6 \cdot 2 \cdot 7 = 84$. \\
    Thus, $m[3, 4] = 84$.\\
    Place value of 3 in $s[3, 4]$ since we split the sequence at $A_3$.
\end{itemize}

\newpage

This gives the updated tables $m$ and $s$.

\begin{center}
\label*{$m$}
\begin{tabular}{ c|c|c|c|c| }
    % Label columns
    \multicolumn{1}{c}{}
    & \multicolumn{1}{c}{1}
    & \multicolumn{1}{c}{2}
    & \multicolumn{1}{c}{3}
    & \multicolumn{1}{c}{4}\\

    % Label rows
    \cline{2-5}
        1 & 0 & 120 &  &  \\
    \cline{2-5}
        2 &  & 0 & 48 &  \\
    \cline{2-5}
        3 &  &  & 0 & 84 \\
    \cline{2-5}
        4 &  &  &  & 0 \\
    \cline{2-5}
\end{tabular}
\end{center}

\begin{center}
\label*{$s$}
\begin{tabular}{ c|c|c|c|c| }
    % Label columns
    \multicolumn{1}{c}{}
    & \multicolumn{1}{c}{1}
    & \multicolumn{1}{c}{2}
    & \multicolumn{1}{c}{3}
    & \multicolumn{1}{c}{4}\\

    % Label rows
    \cline{2-5}
        1 &  & 1 &  &  \\
    \cline{2-5}
        2 &  &  & 2 &  \\
    \cline{2-5}
        3 &  &  &  & 3 \\
    \cline{2-5}
        4 &  &  &  &  \\
    \cline{2-5}
\end{tabular}
\end{center}

%-------------------------------------------------
% Diagonal i = j + 2
%-------------------------------------------------


Now we have calculated the cost of multiplying all pairs possible. Now lets calculate the cost of multiplying sequences with three matrices.

\begin{itemize}
    \item $m[1, 3]$: Consider two possibilities.
    \begin{enumerate}
        \item $A_{5 \times 4} \times ( A_{4 \times 6} \times A_{6 \times 2} ) = A_{5 \times 4} \times A_{4 \times 2}$\\
        Now, get the cost of $A_1$, a single matrix, plus the cost of $A_2 \times A_3$ (the inner multiplication), and getting the overall cost with this specific sequence. Recall $m[1, 1]$ and $m[2, 3]$ have already been calculated.
        \begin{equation*}
            m[1, 1] + m[2, 3] + 5\cdot4\cdot2 = 0 + 48 + 40 = 88
        \end{equation*}
        \item $( A_{5 \times 4} \times A_{4 \times 6} ) \times A_{6 \times 2} = A_{5 \times 6} \times A_{6 \times 2}$
        \begin{equation*}
            m[1, 2] + m[3, 3] + 5\cdot6\cdot2 = 120 + 0 + 60 = 180
        \end{equation*}
        
        Now, take the smaller cost, which is 88, thus $m[1, 3]=88$. Also, $s[1, 3]=1$ since $A_1$ was our cut-off.
    \end{enumerate}
    
    \item $m[2, 4]$: Consider two possibilities again.
    \begin{enumerate}
        \item $A_{4 \times 6} \times ( A_{6 \times 2} \times A_{2 \times 7} ) = A_{4 \times 6} \times A_{6 \times 7}$
        \begin{equation*}
            m[2, 2] + m[3, 4] + 4\cdot6\cdot7 = 0 + 84 + 168 = 252
        \end{equation*}
        \item $( A_{4 \times 6} \times A_{6 \times 2} ) \times A_{2 \times 7} = A_{4 \times 2} \times A_{2 \times 7}$
        \begin{equation*}
            m[2, 3] + m[4, 4] + 4\cdot2\cdot7 = 48 + 0 + 56 = 104
        \end{equation*}
        Now, take the smaller cost, which is 104, thus $m[2, 4]=104$. Also, $s[2, 4]=3$ since $A_3$ separates $A_2 \times A_3$ from $A_4$.
    \end{enumerate}
\end{itemize}

\newpage

That gives us the updated tables.
\begin{center}
\label*{$m$}
\begin{tabular}{ c|c|c|c|c| }
    % Label columns
    \multicolumn{1}{c}{}
    & \multicolumn{1}{c}{1}
    & \multicolumn{1}{c}{2}
    & \multicolumn{1}{c}{3}
    & \multicolumn{1}{c}{4}\\

    % Label rows
    \cline{2-5}
        1 & 0 & 120 & 88 &  \\
    \cline{2-5}
        2 &  & 0 & 48 & 104 \\
    \cline{2-5}
        3 &  &  & 0 & 84 \\
    \cline{2-5}
        4 &  &  &  & 0 \\
    \cline{2-5}
\end{tabular}
\end{center}

\begin{center}
\label*{$s$}
\begin{tabular}{ c|c|c|c|c| }
    % Label columns
    \multicolumn{1}{c}{}
    & \multicolumn{1}{c}{1}
    & \multicolumn{1}{c}{2}
    & \multicolumn{1}{c}{3}
    & \multicolumn{1}{c}{4}\\

    % Label rows
    \cline{2-5}
        1 &  & 1 & 1 &  \\
    \cline{2-5}
        2 &  &  & 2 & 3 \\
    \cline{2-5}
        3 &  &  &  & 3 \\
    \cline{2-5}
        4 &  &  &  &  \\
    \cline{2-5}
\end{tabular}
\end{center}

%-------------------------------------------------
% m[1,4]
%-------------------------------------------------


Now, calculate the cost of the entire sequence, $m[1, 4]$.

\begin{equation*}
    \begin{split}
        m[1, 4] = & \min\{ m[1, 1] + m[2, 4] + 5\cdot4\cdot7,\\
                  & m[1, 2] + m[3, 4] + 5\cdot6\cdot7, \\
                  & m[1, 3] + m[4, 4] + 5\cdot2\cdot7 \\
                = & \min\{ 0+104+140,\\
                  & 120+84+210, \\
                  & 88+0+70 \}\\
                = & \min\{ 244, 414, 158\}\\
        m[1, 4] = & 158
    \end{split}
\end{equation*}

Since our solution comes from splitting $A_1 \cdot A_2 \cdot A_3$ from $A_4$, $s[1, 4] = 3$. This gives our final tables $m$ and $s$.

\begin{center}
\label*{$m$}
\begin{tabular}{ c|c|c|c|c| }
    % Label columns
    \multicolumn{1}{c}{}
    & \multicolumn{1}{c}{1}
    & \multicolumn{1}{c}{2}
    & \multicolumn{1}{c}{3}
    & \multicolumn{1}{c}{4}\\

    % Label rows
    \cline{2-5}
        1 & 0 & 120 & 88 & 158 \\
    \cline{2-5}
        2 &  & 0 & 48 & 104 \\
    \cline{2-5}
        3 &  &  & 0 & 84 \\
    \cline{2-5}
        4 &  &  &  & 0 \\
    \cline{2-5}
\end{tabular}
\end{center}

\begin{center}
\label*{$s$}
\begin{tabular}{ c|c|c|c|c| }
    % Label columns
    \multicolumn{1}{c}{}
    & \multicolumn{1}{c}{1}
    & \multicolumn{1}{c}{2}
    & \multicolumn{1}{c}{3}
    & \multicolumn{1}{c}{4}\\

    % Label rows
    \cline{2-5}
        1 &  & 1 & 1 & 3 \\
    \cline{2-5}
        2 &  &  & 2 & 3 \\
    \cline{2-5}
        3 &  &  &  & 3 \\
    \cline{2-5}
        4 &  &  &  &  \\
    \cline{2-5}
\end{tabular}
\end{center}

\newpage

\subsection*{Solution}
This is the general formula for finding the ideal sequence to multiply matrices $i$ to $j$, with dimensions for rows and columns labeled with $d$.

\begin{equation*}
    m[i, j] = \min_{i \leq k \leq j}\{ m[i, k] + m[k+1, j] + d_{i-1} \cdot d_k \cdot d_j \}
\end{equation*}

Given $n$ matrices, we will have $n+1$ dimensions. Here is the optimal parenthesization.

\begin{equation*}
    (A_1 \cdot A_2 \cdot A_3) \cdot (A_4)
\end{equation*}
Cut off the sequence at third matrix since $s[1, 4] = 3$. \\
\begin{equation*}
    \big((A_1) \cdot (A_2 \cdot A_3)\big) \cdot (A_4)
\end{equation*}
Cut off the sequence at first matrix since $s[1, 3] = 1$.
\\ \\
Thus, this is the sequence of matrices that requires the least number of scalar multiplications.
\begin{equation*}
    \big((A_{5\times4}) \times (A_{4\times6} \times A_{6\times2})\big) \times (A_{2\times7})
\end{equation*}

\end{document}
