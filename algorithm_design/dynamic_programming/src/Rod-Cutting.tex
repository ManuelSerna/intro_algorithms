\documentclass{article}
\usepackage[utf8]{inputenc}

\title{Rod Cutting}
\author{Manuel Serna-Aguilera}
\date{}

\usepackage{natbib}
\usepackage{graphicx}
\usepackage{clrscode3e}
\usepackage{amsmath}

\setlength{\parindent}{0pt}

\begin{document}

\maketitle

\section*{Introduction}
Given a rod of length $n$ units and a table of prices $p_i$ for $i = 1, 2, \ldots, n$, determine the maximum revenue $r_n$ obtainable by cutting up the rod and selling the pieces.
\\ \\
Note that a rod of length n can be cut in $2^{(n-1)}$ different ways.
\\ \\
Next, an example from the book (with the last couple of columns removed for brevity's sake) is used.

\section*{Example}
We are given a table of lengths and associated prices.
\\ \\ \\
\begin{tabular}{ | r | c c c c c c c c |}
\hline
length $i$  & 1 & 2 & 3 & 4 & 5 & 6 & 7 & 8 \\ \hline
price $p_i$ & 1 & 5 & 8 & 9 & 10 & 17 & 17 & 20 \\ \hline
\end{tabular}
\\ \\ \\
Now, with the following formula we can find the optimal cost $r_i$ of the cutting a rod of length $i$, where $p_i$ is the price for a rod of length $i$. 

\begin{equation*}
    r_i = \max\limits_{1 \leq i < j}\{p_i + r_{j-i}\}
\end{equation*}

Now, start the procedure, starting from 

\begin{equation*} \label{rc0}
\begin{split}
r_0 = & 0 \\
\end{split}
\end{equation*}

\begin{equation*} \label{rc1}
\begin{split}
r_1 = & 1 \\
\end{split}
\end{equation*}

\begin{equation*} \label{rc2}
\begin{split}
r_2 = & \{ 1+r_{2-1}=1+r_1=1+1=2, \\
      & 5+r_{2-2}=5+r_0=5+0=5\} \\
    = & 5 \\
\end{split}
\end{equation*}

\begin{equation*} \label{rc3}
\begin{split}
r_3 = & \{ 1+r_{3-1}=1+r_2=1+5=6, \\
      & 5+r_{3-2}=5+r_1=5+1=6, \\
      & 8+r_{3-3}=8+r_0=8+0=8 \}\\
    = & 8 \\
\end{split}
\end{equation*}

\begin{equation*} \label{rc4}
\begin{split}
r_4 = & \{ 1+r_3=1+8=9 \\
      & 5+r_2=5+5=10 \\
      & 8+r_1=8+1=9 \\
      & 9+r_0=9+0=9\} \\
    = & 10
\end{split}
\end{equation*}

Now, simplifying the process to only show the max values.

\begin{equation*} \label{rc5}
\begin{split}
r_5 = & 13\\
\end{split}
\end{equation*}

\begin{equation*} \label{rc6}
\begin{split}
r_6 = & 17\\
\end{split}
\end{equation*}

\begin{equation*} \label{rc7}
\begin{split}
r_7 = & 18\\
\end{split}
\end{equation*}

\begin{equation*} \label{rc8}
\begin{split}
r_8 = & \{ 1+r_{8-1} =1+r_7 = 1+18 =19,\\
    = & 5+r_{8-2} =5+r_6 = 5+17 =22,\\
    = & 8+r_{8-3} =8+r_5 = 8+13 =21,\\
    = & 9+r_{8-4} =9+r_4 = 9+10 =19,\\
    = & 10+r_{8-5}=10+r_3= 10+8 =18,\\
    = & 17+r_{8-6}=17+r_2= 17+5 =22,\\
    = & 17+r_{8-7}=17+r_1= 17+1 =18,\\
    = & 20+r_{8-8}=20+r_0= 20+0 =20 \}\\
    = & 22\\ 
\end{split}
\end{equation*}

So lengths 6 and 2 combined give the highest profit. The max value of $r_2$ and $r_6$ are themselves, so no more cutting. The ordering of the rods of length 2 or 6 does not matter.

\section*{Solving with code}
A regular recursive solution, even for small inputs, may have exponential running time due to repeating many calculations. Instead, use a little more memory to store calculation results. Remember that the bottom-down approach to dynamic programming is often better.


\end{document}
